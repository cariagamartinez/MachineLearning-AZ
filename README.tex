\PassOptionsToPackage{unicode=true}{hyperref} % options for packages loaded elsewhere
\PassOptionsToPackage{hyphens}{url}
%
\documentclass[]{article}
\usepackage{lmodern}
\usepackage{amssymb,amsmath}
\usepackage{ifxetex,ifluatex}
\usepackage{fixltx2e} % provides \textsubscript
\ifnum 0\ifxetex 1\fi\ifluatex 1\fi=0 % if pdftex
  \usepackage[T1]{fontenc}
  \usepackage[utf8]{inputenc}
  \usepackage{textcomp} % provides euro and other symbols
\else % if luatex or xelatex
  \usepackage{unicode-math}
  \defaultfontfeatures{Ligatures=TeX,Scale=MatchLowercase}
\fi
% use upquote if available, for straight quotes in verbatim environments
\IfFileExists{upquote.sty}{\usepackage{upquote}}{}
% use microtype if available
\IfFileExists{microtype.sty}{%
\usepackage[]{microtype}
\UseMicrotypeSet[protrusion]{basicmath} % disable protrusion for tt fonts
}{}
\IfFileExists{parskip.sty}{%
\usepackage{parskip}
}{% else
\setlength{\parindent}{0pt}
\setlength{\parskip}{6pt plus 2pt minus 1pt}
}
\usepackage{hyperref}
\hypersetup{
            pdfborder={0 0 0},
            breaklinks=true}
\urlstyle{same}  % don't use monospace font for urls
\usepackage[margin=1in]{geometry}
\usepackage{graphicx,grffile}
\makeatletter
\def\maxwidth{\ifdim\Gin@nat@width>\linewidth\linewidth\else\Gin@nat@width\fi}
\def\maxheight{\ifdim\Gin@nat@height>\textheight\textheight\else\Gin@nat@height\fi}
\makeatother
% Scale images if necessary, so that they will not overflow the page
% margins by default, and it is still possible to overwrite the defaults
% using explicit options in \includegraphics[width, height, ...]{}
\setkeys{Gin}{width=\maxwidth,height=\maxheight,keepaspectratio}
\setlength{\emergencystretch}{3em}  % prevent overfull lines
\providecommand{\tightlist}{%
  \setlength{\itemsep}{0pt}\setlength{\parskip}{0pt}}
\setcounter{secnumdepth}{0}
% Redefines (sub)paragraphs to behave more like sections
\ifx\paragraph\undefined\else
\let\oldparagraph\paragraph
\renewcommand{\paragraph}[1]{\oldparagraph{#1}\mbox{}}
\fi
\ifx\subparagraph\undefined\else
\let\oldsubparagraph\subparagraph
\renewcommand{\subparagraph}[1]{\oldsubparagraph{#1}\mbox{}}
\fi

% set default figure placement to htbp
\makeatletter
\def\fps@figure{htbp}
\makeatother


\author{}
\date{\vspace{-2.5em}}

\begin{document}

\hypertarget{repositorio-del-curso-machine-learning-de-a-a-la-z-r-y-python-para-data-science}{%
\section{\texorpdfstring{Repositorio del Curso
\href{https://www.udemy.com/draft/2241862/?couponCode=GITHUB_PROMO_JB}{Machine
Learning de A a la Z: R y Python para Data
Science}}{Repositorio del Curso Machine Learning de A a la Z: R y Python para Data Science}}\label{repositorio-del-curso-machine-learning-de-a-a-la-z-r-y-python-para-data-science}}

\hypertarget{creado-por-kirill-eremenko-y-hadelin-de-ponteves}{%
\subsection{\texorpdfstring{Creado por
\href{https://www.udemy.com/user/kirilleremenko/}{Kirill Eremenko} y
\href{https://www.udemy.com/user/hadelin-de-ponteves/}{Hadelin de
Ponteves}}{Creado por Kirill Eremenko y Hadelin de Ponteves}}\label{creado-por-kirill-eremenko-y-hadelin-de-ponteves}}

\hypertarget{traducido-al-espauxf1ol-por-juan-gabriel-gomila-salas}{%
\subsection{\texorpdfstring{Traducido al español por
\href{https://www.udemy.com/user/juangabriel2}{Juan Gabriel Gomila
Salas}}{Traducido al español por Juan Gabriel Gomila Salas}}\label{traducido-al-espauxf1ol-por-juan-gabriel-gomila-salas}}

¿Estás interesado en conocer a fondo el mundo del Machine Learning?
Entonces este curso está diseñado especialmente para ti!!

Este curso ha sido diseñado por Data Scientists profesionales para
compartir nuestro conocimiento y ayudarte a aprender la teoría compleja,
los algoritmos y librerías de programación de un modo fácil y sencillo.

En él te guiaremos paso a paso en el mundo del Machine Learning. Con
cada clase desarrollarás nuevas habilidades y mejorarás tus
conocimientos de este complicado y lucrativa sub rama del Data Science.

\hypertarget{temario-del-curso}{%
\subsubsection{Temario del curso}\label{temario-del-curso}}

Este curso es divertido y ameno pero al mismo tiempo todo un reto pues
tenemos mucho de Machine Learning por aprender. Lo hemos estructurado
del siguiente modo:

\begin{itemize}
\tightlist
\item
  Parte 1 - Preprocesamiento de datos
\item
  Parte 2 - Regresión: Regresión Lineal Simple, Regresión Lineal
  Múltiple, Regresión Polinomial, SVR, Regresión en Árboles de Decisión
  y Regresión con Bosques Aleatorios
\item
  Parte 3 - Clasificación: Regresión Logística, K-NN, SVM, Kernel SVM,
  Naive Bayes, Clasificación con Árboles de Decisión y Clasificación con
  Bosques Aleatorios
\item
  Parte 4 - Clustering: K-Means, Clustering Jerárquico
\item
  Parte 5 - Aprendizaje por Reglas de Asociación: Apriori, Eclat
\item
  Parte 6 - Reinforcement Learning: Límite de Confianza Superior,
  Muestreo Thompson
\item
  Parte 7 - Procesamiento Natural del Lenguaje: Modelo de Bag-of-words y
  algoritmos de NLP
\item
  Parte 8 - Deep Learning: Redes Neuronales Artificiales y Redes
  Neuronales Convolucionales
\item
  Parte 9 - Reducción de la dimensión: ACP, LDA, Kernel ACP
\item
  Parte 10 - Selección de Modelos \& Boosting: k-fold Cross Validation,
  Ajuste de Parámetros, Grid Search, XGBoost
\end{itemize}

Además, el curso está repleto de ejercicios prácticos basados en
ejemplos de la vida real, de modo que no solo aprenderás teoría, si no
también pondrás en práctica tus propios modelos con ejemplos guiados.

Y como bonus, este curso incluye todo el código en Python y R para que
lo descargues y uses en tus propios proyectos.

Puedes apuntarte al curso de Machine Learning con un descuento del 90\%
de su precio original
\href{https://www.udemy.com/draft/2241862/?couponCode=GITHUB_PROMO_JB}{desde
aquí}

\hypertarget{lo-que-aprenderuxe1s}{%
\subsubsection{Lo que aprenderás}\label{lo-que-aprenderuxe1s}}

\begin{itemize}
\tightlist
\item
  Dominar el Machine Learning con R y con Python.
\item
  Tener intuición en la mayoría de modelos de Machine Learning.
\item
  Hacer predicciones precisas y acertadas.
\item
  Crear unos análisis elaborados.
\item
  Crear modelos de Machine Learning robustos y consistentes.
\item
  Crear valor añadido a tu propio negocio.
\item
  Utilizar el Machine Learning para cuestiones personales.
\item
  Dominar aspectos específicos como por ejemplo Reinforcement Learning,
  NLP o Deep Learning
\item
  Conocer las técnicas más avanzadas como la reducción de la
  dimensionalidad.
\item
  Saber qué modelo de Machine Learning usar para cada tipo de problema.
\item
  Crear toda una librería de modelos de Machine Learning y saber cómo
  combinarlos para resolver cualquier problema.
\end{itemize}

\hypertarget{hay-requisitos-para-seguir-correctamente-el-curso}{%
\subsubsection{¿Hay requisitos para seguir correctamente el
curso?}\label{hay-requisitos-para-seguir-correctamente-el-curso}}

\begin{itemize}
\tightlist
\item
  Con el nivel de matemáticas de secundaria y bachillerato es
  suficiente.
\end{itemize}

\hypertarget{para-quiuxe9n-es-este-curso}{%
\subsubsection{¿Para quién es este
curso?}\label{para-quiuxe9n-es-este-curso}}

\begin{itemize}
\tightlist
\item
  Cualquier estudiante que esté interesado en el Machine Learning.
\item
  Estudiantes con nivel de matemáticas de bachillerato que quieren
  iniciarse en Machine Learning.
\item
  Estudiantes de nivel intermedio con conocimientos básicos de Machine
  Learning, incluyendo algoritmos clásicos de regresión lineal o
  logística, pero que quieren aprender más y explorar los diferentes
  campos del Machine Learning.
\item
  Estudiantes que no se sienten cómodos programando pero se interesan
  por el Machine Learning y quieren aplicar las técnicas al análisis de
  data sets.
\item
  Universitarios que quieren iniciarse en el mundo del Data Science.
\item
  Cualquier analista de datos que quiera mejorar sus habilidades en
  Machine Learning.
\item
  Personas que no están satisfechas con su trabajo y quieren convertirse
  en Data Scientist.
\item
  Cualquier persona que quiera añadir valor a su empresa con el poder
  del Machine Learning.
\end{itemize}

\end{document}
